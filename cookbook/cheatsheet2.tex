%Tu si môžete zaznačiť, že pracujete na danej veci. V prípade, že ste napísali len časť a ďalej už
%nechcete, alebo ste hotoví tak sa odtiaľ odpíšte. Bolo by však fajn, aby jedu vec robil jeden
%človek ak celok a zvyšný len kontrolovali
%vypracuva: Zaba 
\input include/include.tex

\begin{document}

\velkynadpis{Ťahák 2}

\nadpis{Triedy a objekty}

Často sa nám pri návrhu hry môže stať, že potrebujeme viacero vecí, ktoré vyzerajú rovnako, správajú
sa rovnako a líšia sa napríklad len pozíciou. Naviac ale každá z týchto vecí chce existovať
samostatne mimo ostatných podobných vecí. Na docielenie práve tohoto dizajnu nám pomáha objektovo
orientovaný návrh.

Ak potrebujeme používať viacero rovnakých vecí, môžeme si navrhnúť šablónu, ktorá bude popisovať ako
sa dané veci správajú. Túto šablónu nazveme \texttt{trieda}. Trieda má dve dôležité časti:
\texttt{vlastnosti}, takže premenné, ktoré popisujú danú vec (napríklad pozícia vo svete, HP, ktoré
pridá) a \texttt{metódy}. Metódy sú funkcie, ktoré daná vec vykonáva nad svojimi vlastnosťami.
Napríklad môžeme mať metódu na pohyb, ktorá posunie príslušnú vec.

Trieda je však len šablóna, ktorá hovorí ako vieme vytvoriť konkrétnu vec. Z triedy vieme vytvárať
jednotlivé \texttt{objekty}, čo sú konkrétne veci v programe. Tie majú svoje vlastné hodnoty vo
vlastnostiach a sú nezávislé od ostatných objektov.

Ďalej si uvedomme, že ak chceme pristupovať k nejakej vlastnosti alebo metóde, tak tieto vlastnosti
\textbf{nepatria triede}, ale \textbf{patria objektu}. Môžeme teda pristupovať len ku vlastnostiam
daného objektu. Tento objekt má svoje meno a keď chceme pristúpiť k jeho vlastnosti alebo chceme nad
ním vykonať metódu, použijeme \texttt{.}(bodka), ktorá spája objekt s jeho vlastnoťou.

Navyše, ak sa chceme v rámci metódy odkazovať na vlastné vlastnosti, použijeme pomocné slovíčko
\texttt{self}. Takisto nezabudnite, že premenné, ktoré má mať daná trieda ako vlastnosti, musia byť
uvedené v špeciálnej funkcii, ktorá sa volá konštruktor. V našom prípade bude mať táto funkcia názov
\texttt{nastav()}.

\nadpis{Trieda v Gaminatore}

Ak chceme vytvoriť triedu v Gaminatore, potrebujeme vedieť niekoľko základných vlastností a metód,
ktoré takáto trieda musí mať. V prvom rade, musí mať každý objekt svoju pozíciu v našom svete. Preto
mu musíme nastaviť vlastnosti $x$ a $y$. Bod $(x,y)$ sa dá následne chápať ako stred našeho objektu.

Najdôležitejšie sú však metódy, ktoré musíme implementovať. Myšlienka celého cyklu triedy je
jednoduchá. Na začiatku objekt \textbf{vznikne} a nastaví si začiatočné vlastnosti. Následne sa bude
v \textbf{každom kroku} programu vykonávať nejaká zmena tohoto objektu (napr. ryba pláve, jedlo padá
\dots) a toto pokračuje, až kým nám je objekt zbytočný a chceme ho \textbf{odstrániť}.

Pri vytvorení objektu sa volá funkcia \texttt{nastav(self)}. Do tejto metódy je potrebné napísať, aké
vlastnosti má daná trieda a aké sú ich začiatočné hodnoty, keď sa bude vytvárať nový objekt. Počas
behu programu sa potom bude volať metóda \texttt{krok(self)}, do ktorej chceme napísať čo má urobiť daný
objekt v každom kroku programu. No a na konci treba zavolať metódu \texttt{znic()}, ktorá zničí daný
objekt. Túto funkciu neprogramujete vy, je už vopred pripravená.

\lstlang{python}\begin{lstlisting}
class Ryba(Vec):
    def nastav(self):
        self.x = 10
        self.y = 10
    
    def krok(self):
        self.y += 1
\end{lstlisting}

\nadpis{Kreslenie objektov}

Objekty sa musia nejakým spôsobom vykresľovať. Na to budeme opäť používať \texttt{kreslic}. Ako však
vieme, každý objekt má nejakú svoju pozíciu $(x,y)$. A táto pozícia sa počas behu programu mení.
Musíme teda meniť aj to, na akých súradniciach sa má vykresľovať konkrétny objekt? Našťastie nie.
Jediné čo potrebujeme je naimplementovať triede metódu \texttt{nakresli(self,kreslic)}.

\end{document}
